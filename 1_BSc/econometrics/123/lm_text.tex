
\documentclass[12pt,a4paper, oneside]{extreport}

%%%%%%%%%% Математика %%%%%%%%%%
\usepackage{amsmath,amsfonts,amssymb,amsthm,mathtools}
% Показывать номера только у тех формул, на которые есть \eqref{} в тексте.
%\mathtoolsset{showonlyrefs=true}
%\usepackage{leqno} % Нумерация формул слева
%\usepackage{tipa} %Для формулки из логитов


\usepackage{hyphenat}

%%%%%%%%%% Шрифты %%%%%%%%
\usepackage[english, russian]{babel} % выбор языка для документа
\usepackage[utf8]{inputenc} % задание utf8 кодировки исходного tex файла


%%%%%%%%%%%%%% ГОСТОВСКИЕ ПРИБАМБАСЫ %%%%%%%%%%%%%%%

%%% размер листа бумаги
\usepackage[paper=a4paper,top=15mm, bottom=15mm,left=35mm,right=10mm,includehead]{geometry}


\usepackage{setspace}
\setstretch{1.5}     % Межстрочный интервал
\setlength{\parindent}{1.25cm} % Красная строка.


%\flushbottom       % Эта команда заставляет LaTeX чуть растягивать строки, чтобы получить идеально прямоугольную страницу
\righthyphenmin=2  % Разрешение переноса двух и более символов
\widowpenalty=10000  % Наказание за вдовствующую строку (одна строка абзаца на этой странице, остальное --- на следующей)
\clubpenalty=10000  % Наказание за сиротствующую строку (омерзительно висящая одинокая строка в начале страницы)
\tolerance=1000     % Ещё какое-то наказание.


% Нумерация страниц сверху по центру
\usepackage{fancyhdr}
\pagestyle{fancy}
%\fancyhead{ } % clear all fields
%\fancyfoot{ } % clear all fields
\fancyhf{}
\fancyhead[R]{}
\fancyfoot[C]{\thepage}
% Чтобы не прорисовывалась черта!
\renewcommand{\headrulewidth}{0pt}


% Нумерация страниц с надписью "Глава"
\usepackage{etoolbox}
\patchcmd{\chapter}{\thispagestyle{plain}}{\thispagestyle{fancy}}{}{}


%%% Заголовки
\usepackage[indentfirst]{titlesec}{\raggedleft}
% Заголовки по левому краю
% опция identfirst устанавливает отступ в первом абзаце



% В Linux этот пакет сделан косячно. Исправляет это следующий непонятный кусок кода.
\makeatletter
\patchcmd{\ttlh@hang}{\parindent\z@}{\parindent\z@\leavevmode}{}{}
\patchcmd{\ttlh@hang}{\noindent}{}{}{}
\makeatother


% Редактирования Глав и названий
\titleformat{\chapter}
{\normalfont\large\bfseries}
{\thechapter }{0.5 em}{}

% Редактирование ненумеруемых глав chapter* (Введение и тп)
\titleformat{name=\chapter,numberless}
{\centering\normalfont\bfseries\large}{}{0.25em}{\normalfont}

% Убирает чеканутые отступы вверху страницы
\titlespacing{\chapter}{0pt}{-\baselineskip}{\baselineskip}

% Более низкие уровни
\titleformat{\section}{\bfseries}{\thesection}{0.5 em}{}
\titleformat{\subsection}{\bfseries}{\thesubsection}{0.5 em}{}

\titlespacing*{\section}{0 pt}{\baselineskip}{\baselineskip}
\titlespacing*{\subsection}{0 pt}{\baselineskip}{\baselineskip}


% Содержание. Команды ниже изменяют отступы и рисуют точечки!
\usepackage{titletoc}

\titlecontents{chapter}
[1em] %
{\normalsize}
{\contentslabel{1 em}}
{\hspace{-1 em}}
{\normalsize\titlerule*[10pt]{.}\contentspage}

\titlecontents{section}
[3 em] %
{\normalsize}
{\contentslabel{1.75 em}}
{\hspace{-1.75 em}}
{\normalsize\titlerule*[10pt]{.}\contentspage}

\titlecontents{subsection}
[6 em] %
{\normalsize}
{\contentslabel{3 em}}
{\hspace{-3 em}}
{\normalsize\titlerule*[10pt]{.}\contentspage}


% Правильные подписи под таблицей и рисунком
% Документация к пакету на русском языке!
\usepackage[tableposition=top, singlelinecheck=false]{caption}
\usepackage{subcaption}


\DeclareCaptionStyle{base}%
[justification=centering,indention=0pt]{}
\DeclareCaptionLabelFormat{gostfigure}{Рисунок #2}
\DeclareCaptionLabelFormat{gosttable}{Таблица #2}

\DeclareCaptionLabelSeparator{gost}{~---~}
\captionsetup{labelsep=gost}

\DeclareCaptionStyle{fig01}%
[margin=5mm,justification=centering]%
{margin={3em,3em}}
\captionsetup*[figure]{style=fig01,labelsep=gost,labelformat=gostfigure,format=hang}

\DeclareCaptionStyle{tab01}%
[margin=5mm,justification=centering]%
{margin={3em,3em}}
\captionsetup*[table]{style=tab01,labelsep=gost,labelformat=gosttable,format=hang}


% межстрочный отступ в таблице
\renewcommand{\arraystretch}{1.2}



% многостраничные таблицы под РОССИЙСКИЙ СТАНДАРТ
% ВНИМАНИЕ! Обязательно за CAPTION !
\usepackage{fr-longtable}



%Более гибкие спсики
\usepackage{enumitem}
% сообщаем окружению о том, что существует такая штук как нумерация русскими буквами.
\makeatletter
\AddEnumerateCounter{\asbuk}{\russian@alph}{щ}
\makeatother


%%% ГОСТОВСКИЕ СПИСКИ

% Первый тип списков. Большая буква.
\newlist{Enumerate}{enumerate}{1}

\setlist[Enumerate,1]{labelsep=0.5em,leftmargin=1.25em,labelwidth=1.25em,
	parsep=0em,itemsep=0em,topsep=0ex, before={\parskip=-1em},label=\arabic{Enumeratei}.}


% Второй тип списков. Маленькая буква.
\setlist[enumerate]{label=\arabic{enumi}),parsep=0em,itemsep=0em,topsep=0.75ex, before={\parskip=-1em}}


% Третий тип списков. Два уровня.
\newlist{twoenumerate}{enumerate}{2}
\setlist[twoenumerate,1]{itemsep=0mm,parsep=0em,topsep=0.75ex,, before={\parskip=-1em},label=\asbuk{twoenumeratei})}
\setlist[twoenumerate,2]{leftmargin=1.3em,itemsep=0mm,parsep=0em,topsep=0ex, before={\parskip=-1em},label=\arabic{twoenumerateii})}


% Четвёртый тип списков. Список с тире.
\setlist[itemize]{label=--,parsep=0em,itemsep=0em,topsep=0ex, before={\parskip=-1em},after={\parskip=-1em}}


%%% WARNING WARNING WARNIN!
%%% Если в списке предложения, то должна по госту стоять точка после цифры => команда Enumerate! Если идет перечень маленьких фактов, не обособляемых предложений то после цифры идет скобка ")" => команда enumerate! Если перечень при этом ещё и двууровневый, то twoenumerate.




%%%%%%%%%% Список литературы %%%%%%%%%%

%\usepackage[%
%backend=biber, %подключение пакета biber (тоже нужен)
%bibstyle=gost-numeric, %подключение одного из четырех главных стилей biblatex-gost
%sorting=ntvy, %тип сортировки в библиографии
%]{biblatex}

\usepackage[backend=biber,style=gost-numeric, maxbibnames=9,maxcitenames=2,uniquelist=false, babel=other]{biblatex}



% Справка по 4 главным стилям для ленивых:
% gost-inline  ссылки внутри теста в круглых скобках
% gost-footnote подстрочные ссылки
% gost-numeric затекстовые ссылки
% gost-authoryear тоже затекстовые ссылки, но немного другие

% Подробнее смотри страницу 4 документации. Она на русском.

% Ещё немного настроек
\DeclareFieldFormat{postnote}{#1} %убирает с. и p.
\renewcommand*{\mkgostheading}[1]{#1} % только лишь убираем курсив с авторов


\addbibresource{diploma11.bib} % сюда нужно вписать свой bib-файлик.



% Этот кусок кода выносит русские источники на первое место. Костыль описали авторы пакета в руководстве к нему. Подробнее смотри:
% https://github.com/odomanov/biblatex-gost/wiki/Как-сделать%2C-чтобы-русскоязычные-источники-предшествовали-остальным
\DeclareSourcemap{
	\maps[datatype=bibtex]{
		\map{
			\step[fieldsource=langid, match=russian, final]
			\step[fieldset=presort, fieldvalue={a}]
		}
		\map{
			\step[fieldsource=langid, notmatch=russian, final]
			\step[fieldset=presort, fieldvalue={z}]
		}
	}
}

\DefineBibliographyStrings{english}{%
	pages = {P\adddot},
	number = {№},
}



\begin{document} % Начала документа



%%%%%%%%%%%%%%%%%%%% ВВЕДЕНИЕ %%%%%%%%%%%%%%%%%%%%%%%%%%%%%%%%%%%%

\subsection*{Задача 1}

\subsection*{1}

$\bar{x} = \dfrac{0.52 + 0.57 + 0.69 + 0.77 + 0.9 + 0.97 + 1.04 + 1.08 + 1.49 + 1.63 }{10} = 0.966$

$\bar{y} = \dfrac{0.28 + 0.33 + 0.34 + 0.34 + 0.33 + 0.38 + 0.46 + 0.49 + 0.52 + 0.49  }{10} = 0.396$

$\sum_{i=1}^n (x_i-\bar{x})(y_i-\bar{y})= (0.52 -0.966)(  0.28 -0.396)+(  0.57 -0.966)(  0.33 -0.396)+(  0.69 -0.966)(  0.34 -0.396)+(  0.77 -0.966)(  0.34 -0.396)+(  0.9 -0.966)(  0.33 -0.396)+(  0.97 -0.966)(  0.38 -0.396)+(  1.04 -0.966)(  0.46 -0.396)+(  1.08 -0.966)(  0.49 -0.396)+(  1.49 -0.966)(  0.52 -0.396)+(  1.63 -0.966)(  0.49 -0.396) = 0.25144$


$\sum_{i=1}^n (x_i-\bar{x})^2 = (0.52 -0.966)^2+( 0.57 -0.966)^2+( 0.69 -0.966)^2+( 0.77 -0.966)^2+( 0.9 -0.966)^2+( 0.97 -0.966)^2+( 1.04 -0.966)^2+( 1.08 -0.966)^2+( 1.49 -0.966)^2+( 1.63 -0.966)^2 = 1.20864 $


$\sum_{i=1}^n (y_i-\bar{y})^2 = (0.28 -0.396)^2+( 0.33 -0.396)^2+( 0.34 -0.396)^2+( 0.34 -0.396)^2+( 0.33 -0.396)^2+( 0.38 -0.396)^2+( 0.46 -0.396)^2+( 0.49 -0.396)^2+( 0.52 -0.396)^2+( 0.49 -0.396)^2 = 0.06584$

$\hat{\rho} = \dfrac{0.25144}{\sqrt{ 0.32844 * 2.39184} } = 0.891335030907108$

$ \hat{\rho}  > 0   $, следовательно,  между переменными положительная взаимосвязь.  

$0.7 < \hat{\rho}  < 0.9  $, следовательно, по шкале Чеддока между переменными корреляция высокая.  

\subsection*{2}

$R^2 = \hat{\rho}^2 = 0.891335030907108^2= 0.794478137322176 $

79,4 \% дисперсии результативного признака $y$ объясняется влиянием факторного признака $x$.


$$y_i = \beta_1 + \beta_2 x_i + e_i  $$


$\hat{\beta}_2 = \dfrac{Cov(x,y)}{Var(x)} = \dfrac{0.25144}{1.20864} = 0.208035477892507 $


$\hat{\beta}_1 = \bar{y} - \hat{\beta_2}  \bar{x}  = 0.396 -  0.208035477892507 * 0.966 =  0.195037728355838$


$$y_i = 0.195 + 0.208 x_i + e_i  $$

В среднем с ростом з/п на 1 тыс. руб. прожиточный минимум на душу населения в увеличивается на 208 руб.


\subsection*{4}

$F_{obs} = \dfrac{R^2}{1-R^2} (n-2) = \dfrac{0.794478137322176}{1-0.794478137322176} (10-2) = 30.9252992152021$

$F_{crit} = F_{1, 8, 0.05}  = 5.31765507157871$

$F_{obs} > F_{crit}$ коэффициент детерминации является статистически значимым, следовательно, рассматриваемая регрессия достаточно определена включенными переменными.  

\subsection*{5}

$s^2_{e} = \dfrac{\sum^n_{i=1}e_i^2}{n-2}=({-0.0232161768599419 }^2+{ 0.016382049245433 }^2+{ 0.00141779189833205 }^2-{ 0.0152250463330685 }^2-{ 0.0522696584590945 }^2-{ 0.01683214191157 }^2+{ 0.0486053746359545 }^2+{ 0.0702839555202541 }^2+{ 0.0149894095843262 }^2-{ 0.0441355573206249 }^2):8 = 0.0016914449298385$

$se_{\hat{\beta}_1} = \dfrac{s^2_e * \sum_{i=1}^n x_i^2 }{n* \sum_{i=1}^n (x_i-\bar{x})^2 }	= \sqrt{\dfrac{0.0016914449298385 * 10.5402}{1.20864 * 10} }= 0.0384065127359383 $

$se_{\hat{\beta}_2} = \sqrt{\dfrac{s^2_e}{\sum_{i=1}^n (x_i-\bar{x})^2 }}= \sqrt{\dfrac{0.0016914449298385}{1.20864}} = 0.0374093747612531 $

$t_{\beta_1} = \dfrac{\hat{\beta}_1}{se_{\hat{\beta}_1}} = \dfrac{0.195037728355838}{0.0384065127359383} = 5.07824622601923$

$t_{\beta_2} = \dfrac{\hat{\beta}_2}{se_{\hat{\beta}_2}} = \dfrac{0.208035477892507}{0.0374093747612531} = 5.56105198817652$

$t_{crit} = t_{0.975, 8} =  2.30600413520417 $

$|t_{\beta_2}| > |t_{\beta_1}| > t_{crit} $, следовательно, гипотеза о незначимости коэффициента отвергается в обоих случаях на 5\% уровне значимости. 

\subsection*{6}

$x_0=1.04$

$\hat{y_0} = 0.195037728355838 + 0.208035477892507 * 1.04 = 0.411394625364045  $

$s_{\hat{y_0}}^2 = s^2_e * \left(1  + \dfrac{1}{n} + \dfrac{(x_0-\bar{x})^2}{\sum_{i=1}^n (x_i-\bar{x})^2 } \right) = 0.0016914449298385^2 * $

$ \left(1  + \dfrac{1}{10} + \dfrac{(1.04-0.966)^2}{1.20864 } \right)   = 0.0454262333504169$

$e_0 = y_0 - \hat{y_0} = 0.46 - 0.411394625364045  = 0.04860537463 $

95\% доверительный интервал прогноза индивидуального значения:

$(\hat{y_0} + t_{0.025, 8} * s_{\hat{y_0}}; \hat{y_0} + t_{0.975, 8} * s_{\hat{y_0}}) = (0.411394625364045 -  2.30600413520417 * \sqrt{0.04542623} ; $

$ 0.411394625364045 +  2.30600413520417 * \sqrt{0.04542623} ) = (0.411394625364045 -  0.0996730874667656 ;  0.411394625364045 +  0.0996730874667656  ) =$

$= (0.311721512533234; 0.511067687466766) $

Согласно оцененной модели 
при средней з/п в 1.04 тыс. руб. прожиточный минимум на душу населения равен 411.4 руб.


\subsection*{7}


$x_0=9$

$\hat{y_0} = 0.195037728355838 + 0.208035477892507 * 9 = 2.06735702939 $

$s_{M}^2 = s^2_e * \left( \dfrac{1}{n} + \dfrac{(x_0-\bar{x})^2}{\sum_{i=1}^n (x_i-\bar{x})^2 } \right) = 0.0016914449298385^2 *$

$ \left( \dfrac{1}{10} + \dfrac{(9-0.966)^2}{1.20864 } \right)   = 0.0904975937101495$

$s_{\hat{y_0}}^2 = s^2_e * \left(1  + \dfrac{1}{n} + \dfrac{(x_0-\bar{x})^2}{\sum_{i=1}^n (x_i-\bar{x})^2 } \right) = 0.0016914449298385^2 *$

$ \left(1  + \dfrac{1}{10} + \dfrac{(9-0.966)^2}{1.20864 } \right)   = 0.092189038639988$

95\% доверительный интервал:

$(E(y_0|x_0=9) + t_{0.025, 8} * s_{M}^2 ;E(y_0|x_0=9) + t_{0.975, 8} * s_{M}^2  ) = ( 2.06735702939  -  2.30600413520417 * \sqrt{0.0904975937101495} ; $

$  2.06735702939 +  2.30600413520417 * \sqrt{0.0904975937101495} ) = $

$= (1.373646; 2.761068) $


$(\hat{y_0} + t_{0.025, 8}* s_{\hat{y_0}}; \hat{y_0} + t_{0.975, 8}* s_{\hat{y_0}} ) = ( 2.06735702939 -  2.30600413520417 * \sqrt{0.092189038639988} ; $

$  2.06735702939  +  2.30600413520417 * \sqrt{0.092189038639988} ) = $

$= (1.367193; 2.767521) $

Доверительный интервал для индивидуального прогноза шире, чем для условного математического ожидания, так как в первом также учиитывается дополнительно отклонения значений $y_0$ от своего математического ожидания $E(y_0| x = 9)$.


\section*{Задача 2}

\subsection*{1}



\begin{equation}\label{key}
X = 
\begin{bmatrix} 
1 & 49 & 16 \\ 
1 & 20 & 44 \\ 
1 & 31.9 & 13 \\ 
1 & 33.4 & 12 \\ 
1 & 35.3 & 12 \\ 
1 & 24.6 & 18 \\ 
1 & 30.8 & 22 \\ 
1 & 43.4 & 8 \\ 
1 & 42.4 & 10 \\ 
1 & 53.8 & 7 \\ 
1 & 60.6 & 7 \\ 
1 & 58.1 & 6 \\ 
\end{bmatrix}
\end{equation}

\begin{equation}\label{key}
Y=\begin{bmatrix} 
71 \\ 67 \\ 72 \\ 71 \\ 72 \\ 73 \\ 73 \\ 78 \\ 72 \\ 77 \\ 76 \\ 77\end{bmatrix}
\end{equation}

\begin{equation}\label{key}
X^TX=\begin{bmatrix} 
12 & 483.3 & 175 \\ 
483.3 & 21357.79 & 5944.1 \\ 
175 & 5944.1 & 3755 \\ 
\end{bmatrix}
\end{equation}

\begin{equation}\label{key}\
(X^TX)^{-1} = 
\begin{bmatrix} 
3.53352 & -0.06100303 & -0.06811128 \\ 
-0.06100303 & 0.001136855 & 0.001043395 \\ 
-0.06811128 & 0.001043395 & 0.00178893 \\ 
\end{bmatrix}
\end{equation}

\begin{equation}\label{key}
X^TY = \begin{bmatrix} 
879 \\ 
35732.9 \\ 
12533 \\ 
\end{bmatrix}
\end{equation}

\begin{equation}\label{key}
\hat{\beta} = \begin{bmatrix} 
\beta_0 \\ 
\beta_1 \\ 
\beta_2 \\ 
\end{bmatrix}  = (X^TX)^{-1}X^TY = \begin{bmatrix} 
72.51012 \\ 
0.07834808 \\ 
-0.1656405 \\ 
\end{bmatrix}
\end{equation}

Уравнение регрессии имеет вид: 

$$y_i = \beta_0 + \beta_1 x_{1i}  + \beta_2 x_{2i} + e_i  $$


$$\hat{y}_i = \hat{\beta}_0 + \hat{\beta}_1 x_{1i}  + \hat{\beta}_2 x_{2i}  =  72.51012  + 0.07834808  *  x_{1i} -0.1656405  x_{2i}  $$


Константа в регрессии отражает усредненное влияние всех неучтенных факторов. В среднем при прочих равных при увеличении $x_1$ на 1 значение $y$ вырастет на 
0.07834808, т.е. $x_1$ положительно влияет на $y$.   В среднем при прочих равных при увеличении $x_2$ на 1 значение $y$ снижается на 
0.1656405, т.е. $x_1$ отрицательно  влияет на $y$.  


\subsection*{2}

$e = y - \beta^T*x = \begin{bmatrix} 
-2.69893 \\ 0.2110998 \\ -0.856099 \\ -2.139262 \\ -1.288123 \\ 1.544045 \\ 1.720849 \\ 3.414695 \\ -2.175675 \\ 1.434235 \\ -0.09853209 \\ 0.9316976\end{bmatrix} $

$\hat{\sigma}_e = e^T*e * \dfrac{1}{12-1-2}  =  4.330144 $


\begin{equation}\label{key}
\widehat{Var}(\hat{\beta})  =   \hat{\sigma}_e * (X^T*X)^{-1}  =  4.330144 * 
\begin{bmatrix} 
	3.53352 & -0.06100303 & -0.06811128 \\ 
	-0.06100303 & 0.001136855 & 0.001043395 \\ 
	-0.06811128 & 0.001043395 & 0.00178893 \\ 
\end{bmatrix} =
\end{equation}

\begin{equation}\label{key}
= \begin{bmatrix} 
15.30065 & -0.2641519 & -0.2949316 \\ 
-0.2641519 & 0.004922747 & 0.004518049 \\ 
-0.2949316 & 0.004518049 & 0.007746324 \\ 
\end{bmatrix}
\end{equation}

$se_{\hat{\beta}_0} =  \sqrt{\widehat{Var}(\hat{\beta})_{11} } =  \sqrt{15.30065}=  3.911604$

$se_{\hat{\beta}_1} =  \sqrt{\widehat{Var}(\hat{\beta})_{22} } =  \sqrt{0.004922747 }=  0.07016229 $

$se_{\hat{\beta}_2} =  \sqrt{\widehat{Var}(\hat{\beta})_{33} } =  \sqrt{0.007746324}= 0.08801320  $


$t_{\beta_1} = \dfrac{\hat{\beta}_0}{se_{\hat{\beta}_0}} = \dfrac{72.51012}{3.91160} = 18.537 $


$t_{\beta_1} = \dfrac{\hat{\beta}_1}{se_{\hat{\beta}_1}} = \dfrac{0.07835 }{ 0.07016} = 1.117$

$t_{\beta_2} = \dfrac{\hat{\beta}_2}{se_{\hat{\beta}_2}} = \dfrac{ -0.16564}{0.08801 } = -1.882 $

$t_{crit} = t_{0.975, 9} = 2.2621571627982 $

$|t_{\beta_0}| >  t_{crit} $, следовательно, гипотеза о незначимости константы отвергается на 5\% уровне значимости. 

$|t_{\beta_1}| <  t_{crit} $, следовательно, гипотеза о незначимости коэффициента не отвергается на 5\% уровне значимости. 

$|t_{\beta_2}| <  t_{crit} $, следовательно, гипотеза о незначимости коэффициента не отвергается на 5\% уровне значимости. 


\subsection*{3}

95\% доверительные интервал для $\beta_0$:

$(\hat{\beta_0} + t_{0.025, 9} * s_{\hat{\beta_0}};\hat{\beta_0} + t_{0.975, 9} * s_{\hat{\beta_0}} = (72.51012 -  2.2621571627982 * 3.91160 ; 72.51012 +  2.2621571627982 * 3.91160) = (63.66145876;	81.3587859)$


95\% доверительные интервал для $\beta_1$:

$(\hat{\beta_1} + t_{0.025, 9} * s_{\hat{\beta_1}};\hat{\beta_1} + t_{0.975, 9} * s_{\hat{\beta_1}} = ( 0.07835  -  2.2621571627982 * 0.07016  ; 0.07835  +  2.2621571627982 * 0.07016 ) = (-0.08037006;	0.23706621)$

95\% доверительные интервал для $\beta_2$:

$(\hat{\beta_2} + t_{0.025, 9} * s_{\hat{\beta_2}};\hat{\beta_2} + t_{0.975, 9} * s_{\hat{\beta_2}} = (-0.16564 -  2.2621571627982 *   0.08801  ; -0.16564 +   2.2621571627982 *   0.08801 ) = (-0.36474024;	0.03345916)$


0 принадлежит 95-\% доверительному интервалу для $\beta_0$ --  коэффициент значимы на 5\% уровне.
0 принадлежит 95-\% доверительному интервалу для $\beta_1$, $\beta_2$ -- эти коэффициенты незначимы на 5\% уровне.


\subsection*{4}

$R^2 = 1 - \dfrac{e^Te}{(Y-\bar{Y})^T(Y-\bar{Y})} =   1 - \dfrac{38.9712916989129}{112.25}  = 0.652817000455118$

65,3 \% дисперсии результативного признака $y$ объясняется влиянием факторных признаков $x_1$, $x_2$.


\subsection*{5}

$F_{obs} = \dfrac{R^2 * (n-1-2) }{(1-R^2) 2}  = \dfrac{0.652817000455118 * 9}{(1-0.652817000455118)*2}  = 8.46146414397881$

$F_{crit} = F_{2, 9, 0.05}  = 4.25649472909375$

$F_{obs} > F_{crit}$ коэффициент детерминации является статистически значимым, следовательно, рассматриваемая регрессия достаточно определена включенными переменными.  


\subsection*{6}

Проведем тест ранговой корреляции Спирмена для переменной $x_1$:




\begin{equation}\label{key}
d_1 = r_e-r_{x1}  =  \begin{bmatrix} 
1 \\ 7 \\ 5 \\ 3 \\ 4 \\ 10 \\ 11 \\ 12 \\ 2 \\ 9 \\ 6 \\ 8\end{bmatrix} -  \begin{bmatrix} 
9 \\ 1 \\ 4 \\ 5 \\ 6 \\ 2 \\ 3 \\ 8 \\ 7 \\ 10 \\ 12 \\ 11\end{bmatrix}=  \begin{bmatrix} 
-8 \\ 6 \\ 1 \\ -2 \\ -2 \\ 8 \\ 8 \\ 4 \\ -5 \\ -1 \\ -6 \\ -3\end{bmatrix}
\end{equation}

$\rho_1 = 1 -\dfrac{6*d^T_1*d}{n(n^2-n)} = 1 -\dfrac{6*324}{12(12^2-12)} = -0.132867132867133 $

$t_{obs} = \dfrac{\rho_1 \sqrt{n-2} }{\sqrt{1-\rho^2_1} } = \dfrac{-0.132867132867133 \sqrt{12-2} }{\sqrt{1-0.132867132867133^2} } = -0.423921312476319 $

$t_{crit}  = t_{0.975, 10} = 2.22813885198627$

$|t_{obs}| < t_{crit} $, следовательно, нулевая гипотеза об отстутствии корреляции между вектором ошибок и $x_1$ не отвергается.   
 
Проведем тест ранговой корреляции Спирмена для переменной $x_2$:

\begin{equation}\label{key}
d_2 = r_e-r_{x2}  =  \begin{bmatrix} 
1 \\ 7 \\ 5 \\ 3 \\ 4 \\ 10 \\ 11 \\ 12 \\ 2 \\ 9 \\ 6 \\ 8\end{bmatrix} -  \begin{bmatrix} 
9 \\ 12 \\ 8 \\ 6.5 \\ 6.5 \\ 10 \\ 11 \\ 4 \\ 5 \\ 2.5 \\ 2.5 \\ 1\end{bmatrix} =   \begin{bmatrix} 
-8 \\ -5 \\ -3 \\ -3.5 \\ -2.5 \\ 0 \\ 0 \\ 8 \\ -3 \\ 6.5 \\ 3.5 \\ 7\end{bmatrix}
\end{equation}

$\rho = 1 -\dfrac{6*d^T_2*d}{n(n^2-n)} = 1 -\dfrac{6*293}{12(12^2-12)} = -0.0244755244755244 $

$t_{obs} = \dfrac{\rho_1 \sqrt{n-2} }{\sqrt{1-\rho^2_1} } = \dfrac{-0.0244755244755244 \sqrt{12-2} }{\sqrt{1-0.0244755244755244^2} } = -0.0774215974981098 $

$t_{crit}  = t_{0.975, 10} = 2.22813885198627$

$|t_{obs}| < t_{crit} $, следовательно, нулевая гипотеза об отстутствии корреляции между вектором ошибок и $x_2$ не отвергается.   


Вывод -- ошибки гомоскедастичны.

\newpage

\subsection*{7}


$DW = \dfrac{\sum_2^{12} (e_t - e_{t-1})^2}{\sum_1^{12} e_t^2} = (( 0.211099759340403 + 2.6989295510695 )^2+ ( -0.856099032445978 - 0.211099759340403 )^2+ ( -2.13926168682379 + 0.856099032445978 )^2+ ( -1.28812303475612 + 2.13926168682379 )^2+ ( 1.54404462387122 +1.28812303475612 )^2+ ( 1.72084869150117 - 1.54404462387122 )^2+ ( 3.41469538422851 - 1.72084869150117 )^2+ ( -2.17567546273412 - 3.41469538422851 )^2+ ( 1.43423483690452 +2.17567546273412 )^2+ ( -0.0985320925375037 - 1.43423483690452 )^2+ ( 0.931697564521175 +0.0985320925375037 )^2)/(( -2.6989295510695 )^2+ ( 0.211099759340403 )^2+ ( -0.856099032445978 )^2+ ( -2.13926168682379 )^2+ ( -1.28812303475612 )^2+ ( 1.54404462387122 )^2+ ( 1.72084869150117 )^2+ ( 3.41469538422851 )^2+ ( -2.17567546273412 )^2+ ( 1.43423483690452 )^2+ ( -0.0985320925375037 )^2+ ( 0.931697564521175 )^2)= \dfrac{70.5941233789183}{38.9712916989129} = 1.81143914664977$

$k = 2, n=12$

$ d_{U,0.05} = 1.274 $

$ d_{L,0.05} = 0.569 $

$DW > d_U$, следовательно, гипотеза об отсутствии автокорреляции не отвергается.


\end{document}
