%!TEX TS-program = xelatex
% Исходная версия шаблона --- 
% https://www.writelatex.com/coursera/latex/5.1
\documentclass[c, dvipsnames]{beamer}  % [t], [c], или [b] --- вертикальное 
%\documentclass[handout, dvipsnames, c]{beamer} % Раздаточный материал (на слайдах всё сразу)
\input{pre}
\title[R\&D Investment, Exporting, and \\	Productivity Dynamics]{R\&D Investment, Exporting, and \\	Productivity Dynamics}
%\subtitle{Защита выпускной квалификационной работы}


 \usepackage{amsmath}

\author[Касьянова Ксения]{Касьянова Ксения \\ \smallskip \scriptsize ЭО-15-01 }


%\author[Имя автора]{Имя автора \\ \smallskip \scriptsize \href{mailto:author@ranepa.ru}{author@ranepa.ru} \\ \smallskip  \href{http://ranepa.ru}{http://ranepa.ru} }

\institute[РАНХиГС]{ \uppercase{
  Российская Академия Народного Хозяйства и  \\ Государственной Службы при Президенте Российской Федерации}}
\date{}


\titlegraphic{\includegraphics[scale=0.5]{logo1}}
\titlegraphicii{\includegraphics[scale=0.5]{logo2}}

\begin{document}

\frame[plain]{\titlepage}	% Титульный слайд




\section{Motivation:}

\begin{frame}[shrink=3]
\frametitle{\insertsection} 
\begin{block}{Research question:}
	\begin{itemize}
		%		\item  Улучшить прогноз ВВП России с помощью иерархических моделей. 
		\item   Does openness to trade promote productivity?
	\end{itemize}
	
\end{block}


\begin{block}{Motivation:}



Exports and firm productivity are correlated
\begin{itemize}
	\item   Selection: robust finding
	\item  Learning-by-Exporting: mixed evidence
	
\end{itemize} 


Interdependent R\&D and exporting at firm-level
\begin{itemize}
	\item   Exporting → larger market → more R\&D incentive
	\item  Selection
	\item R\&D → higher expected future productivity → re-inforce
	selection
	
\end{itemize}

Market size effect depend on
\begin{itemize}
	\item   domestic/exporting profitability
	\item  innovation process
	\item  costs assoicated with each activity
	
\end{itemize}

\end{block}



\end{frame}


%\begin{frame}[shrink=3]
%\frametitle{\insertsection} 
%
%
%
%
%• Document firm-level R\&D and export dynamics in Taiwanese
%Electronics Industry.
%
%• Estimate an empirical structural model and quantify how optimal R\&D and export decisions depend on expectation of future productivity/export demand; fixed/sunk costs assoicated with choices
%
%• Decompose correlated R\&D and Export dynamics by
% selection on unobservables;
% state dependence
%
%• To be done: quantify how large is the market size effect, i.e. how
%does R\&D respond to trade cost reduction?
%
%\end{frame}
%
%
%


\section{Literature}


\begin{frame}[shrink=3]
\frametitle{\insertsection} 

\begin{enumerate}
	\item Empirics on exporting and productivity: Bernard and Jensen (1999); Bustos (2007)
\item  Recent macro/trade models of joint decisions of R\&D and export:
Atkeson and Burstein (2008); Constantini and Melitz (2008)
\item  Structural estimation of industry equilibrium:
Olley and Pakes (1996): productivity dynamics;
Das, Roberts and Tybout (2007): exporting with sunk and
fixed costs

\item  A reflection of a new mechanism: endogenous innovation?
 Bustos (2007), Lileeva and Trefler (2007): trade liberalization induces more innovation;
 Aw, Roberts, and Winston (2007): R\&D and exporting
correlated.

\end{enumerate}


\end{frame}




\section{Theoretical Model}





\begin{frame}[shrink=3]
\frametitle{\insertsection} 

\textbf{Technology:}

• Short-run marginal cost:

$$lnc_{it} = lnc(k_{it},w_{t}) − x_{it} = \beta_0 + \beta_{k }lnk_{it} +\beta_{w} lnw_{t} − x_{it}$$

• $k_{it }$ capital stock, $w_{t}$ variable input price, $x_{it}$ productivity
• Differs across firms, but not a function of output
• Two sources of heterogeneity: capital-observable, productivity-not
observable by researchers.

\end{frame}


\begin{frame}[shrink=3]
\frametitle{\insertsection} 


\textbf{Demand:}

• Demand for the firm’s output in domestic market (Dixit -Stiglitz)
$$q^D_{it} = Q^D_t(p^D_{it} /P^D_t)^{\eta_D} = \dfrac{I^D_t}{P^D_t} \left( \dfrac{p^D_{it}}{P^D_t} \right)^{\eta_D} = \Phi^D_t (p^D_{it} )^{\eta_D}$$ 

• All aggregates are combined into $\Phi^D_t$

• Similarly, demand for the firm’s output in export market:

$$q^X_{it} = Q^X_t(p^X_{it} /P^X_t)^{\eta_X} = z_{it} \dfrac{I^X_t}{P^X_t} \left( \dfrac{p^X_{it}}{P^X_t} \right)^{\eta_X} = \Phi^X_t z_{it} (p^X_{it} )^{\eta_X}$$ 


• $z_{it}$: firm-specific demand shock in export market. Heterogeneity
between export and domestic market for each firm.

\end{frame}

\begin{frame}[shrink=3]
\frametitle{\insertsection} 

• These assumptions imply domestic and export revenue function as:

$$ln r^D_{it} = (\eta_D + 1) ln( \frac{\eta_D}{\eta_D + 1}) + ln \Phi^D_t + (\eta_D + 1)lnc_{it}$$

$$ln r^X_{it} = (\eta_X + 1) ln( \frac{\eta_X}{\eta_X + 1}) + ln \Phi^X_t + (\eta_X + 1)lnc_{it} + z_{it}$$

• Profits: directly relate revenue to unobservables $x_{it}$ and $z_{it}$.

$$\pi^D_{it} = (−1/\eta_D)r^D_{it} (\Phi^D_t, k_{it}, x_{it})$$

$$\pi^X_{it} = (−1/\eta_X)r^X_{it} (\Phi^X_t, k_{it}, x_{it}, z_{it})$$



Finally, total cost $tvc_{it} = r^D_{it} (1 + \frac{1}{\eta_D}) + r^X_{it} (1 + \frac{1}{\eta_X})$

\end{frame}




\begin{frame}[shrink=3]
\frametitle{\insertsection} 

• Productivity $x_{it}$ evolves endogenously, depending on R\&D $d_{it−1}$
and exporting $e_{it−1}$:
$$x_{it} = g(x_{it−1}, d_{it−1}, e_{it−1}) + \psi_{it}$$

• $d_{it−1}$: learning-by-investing. $e_{it−1}$: learning-by-exporting. $d, e$: discrete (0/1) or continuous.

• Export demand shock $z_{it}$ evolves exogenously as a first order markov process:
$$z_{it} = \rho_z z_{it−1} + \mu_{it}, \mu_{it} \sim N(0, \sigma^2_{\mu}$$

• Firm size measure capital $k_i$: short time series dimension with very little variation over time


\end{frame}





\begin{frame}[shrink=3]
\frametitle{\insertsection} 

\textbf{Sources of dynamics:}

• $e$ and $d$ affect evolution of future $x$; $z$ is persistent over time

• Beginning each activity involves one-time sunk cost.

Sequence of Information and Decisions:

\begin{enumerate}
	\item  Begin period t with productivity and export demand shock $(x_{it}, z_{it})$.
\item  Random fixed cost $\gamma^F_{it}$ of exporting and sunk cost $γ^S_{it} \to$  export decision.
\item  Maximize static profits $\pi^D_{it}$ and, if exporting, $\pi^X_{it}$.
\item  Random fixed cost of R\&D $\gamma^I_{it}$ and sunk cost $\gamma^D_{it} \to$ R\&D decision.
\item  End of period t, new states $(x_{it+1}, z_{it+1})$ realized.

\end{enumerate}
\end{frame}





\begin{frame}[shrink=3]
\frametitle{\insertsection} 


\begin{figure}
	\centering
	\includegraphics[width=0.7\linewidth]{screenshot004}


	\label{fig:screenshot004}
\end{figure}



\end{frame}




\begin{frame}[shrink=3]
\frametitle{\insertsection} 


Firm’s problem:


$$ \max_{\{e_{t},d_{t}\}} \left\{ E_0 \sum^\infty_{t=0} \delta^t \left\{ \pi^D(\omega_t) +e_t [\pi^X(\omega_t,z_t)-\gamma^X(e_{t-1})] -d_t\gamma^R(d_{t-1})\right\}\right\}$$

subject to productivity evolution:

$$\omega_t = g(\omega_{t−1}, e_{t−1}, d_{t−1})$$


\begin{itemize}
	\item No static optimization
\item 
High $\omega_t$ affects incentives for both $e_t$ and $d_t$
\item 
Interactions between $e_t$ and $d_t$ through both objective
function and productivity dynamics
\item 
Persistence through sunk versus fixed costs (both iid): option
value of waiting


\end{itemize}
\end{frame}



\begin{frame}[shrink=3]
\frametitle{\insertsection} 


\begin{figure}
	\centering
	\includegraphics[width=0.7\linewidth]{screenshot005}
	
	
	\label{fig:screenshot005}
\end{figure}



\end{frame}








\section{Estimation}




\begin{frame}[shrink=3]
\frametitle{\insertsection} 


Static equations:

\begin{itemize}
	\item $\{tvc_{it},r^D_{it},r^X_{it}\}$ to estimate elasticity of demand
\item 
$\{r^D_{it}, k_{it}, m_{it}, n_{it}\}$ to estimate productivity $\omega_{it}$
\item 
$ \{r^X_{it}, \omega_{it}\}$ to estimate export demand shock $z_{it}$
\end{itemize}

Productivity dynamics:
$$\omega_{it} = g(\omega_{it−1}, e_{it−1}, d_{it−1})$$
Estimated by OLS using a parametric assumption about $g(·)$

Dynamic exporting and investment decisions:

$\{e_{it}, d_{it}|z_{it}\}$ to estimate parameters of the model (sunk and fixed costs) using ML

\end{frame}




\begin{frame}[shrink=3]
\frametitle{\insertsection} 

\begin{figure}
	\centering
	\includegraphics[width=0.7\linewidth]{screenshot006}
	\label{fig:screenshot006}
\end{figure}

\end{frame}




\begin{frame}[shrink=3]
\frametitle{\insertsection} 

\begin{figure}
	\centering
	\includegraphics[width=0.7\linewidth]{screenshot007}
	\label{fig:screenshot007}
\end{figure}

\begin{figure}
	\centering
	\includegraphics[width=0.7\linewidth]{screenshot009}
	\label{fig:screenshot009}
\end{figure}


\end{frame}


\section{Data}

\begin{frame}[shrink=3]
\frametitle{\insertsection} 

Taiwanese Electronics Industry

• Balanced panel of 1,237 plants for 2000-2004


• Product classes: consumer electronics, telecommunication
equipment, computers and storage equipment, electronics parts and
components.

• Most dynamic industry in Taiwanese manufacturing sector
\begin{itemize}
	\item   Export participation .39 - compete with low-margins
	\item  R\&D performers .17 - major focus on process innovation
	\item  Significant cross-sectional heterogeneity in productivity and
	activities.
	\item  Sustained productivity growth, 3.6\% annual in 80s and 90s.
	
\end{itemize}

Key variables: Revenue-domestic and export, Physical capital stocks
(size), R\&D expenditure, Variable costs-material, labor, energy


\end{frame}


\begin{frame}[shrink=3]
\frametitle{\insertsection} 


• Transition pattern of R\&D and exporting:

\begin{figure}
\centering
\includegraphics[width=0.7\linewidth]{screenshot002}
\label{fig:screenshot001}
\end{figure}


• Persistence in the status: (1) high sunk costs (2) high degree of
persistence in the underlying profit heterogeneity.

• Exporting is more common than R\&D investment.

• Undertaking one of the activities in year t→ more likely to add the
other in year t + 1, less likely to drop the other in year t + 1


\end{frame}


\section{Estimation}


\begin{frame}[shrink=3]
\frametitle{\insertsection} 

\begin{figure}
	\centering
	\includegraphics[width=0.7\linewidth]{screenshot008}
	\label{fig:screenshot008}
\end{figure}


\end{frame}






\section{Results}




\begin{frame}[shrink=3]
\frametitle{\insertsection} 


Productivity dynamics (estimation of $g(·)$):
$$\dfrac{ \delta \omega_{it}>0}{\delta e_{it-1}} > 0; \dfrac{\delta\omega_{it}}{\delta_{it−1}}> 0; \dfrac{\delta^2\omega_{it}}{\delta e_{it}\delta d_{it}}< 0$$

Sunk and Fixed costs of Exporting and R\&D:
\begin{itemize}
	\item  R\&D costs roughly twice as big as Export costs
\item  Sunk costs are roughly twice as big as Fixed costs
\item  Around 10\% of revenues

\end{itemize}

Interdependence between exporting and investment:
\begin{itemize}
	\item  Selection based on $\omega_{it}$ for both $e_{it}$ and $d_{it}$
\item  A lot of persistence due to large sunk costs relative to fixed
costs
\item  Probability of exporting decreasing in R\&D and probability of
investment decreases in export status due to the interaction in
the productivity dynamics

\end{itemize}


\end{frame}


\begin{frame}[shrink=3]
\frametitle{\insertsection} 

• Productivity in response to both R\&D and exporting.

• Impact of R\&D is larger.

• But, relatively low exporting cost makes it a more important
channel.

• The interdependence of R\&D and exporting is dominated by
selection: stable export demand.



\end{frame}


\end{document}